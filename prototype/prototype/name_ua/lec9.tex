\section{Доповнення про нормовані векторні простори}
\subsection{Послідовності функцій}
$X$ множина ($X \subset \R$), $f_n: X \to \R(\mathbb{C})$ і $(f_n)_{n \in \N}$. Корисно для подальшої частини розділу: $B(X, \R)$ позначає множину функцій $f: X \to \R$ \underline{обмежених}
\subsection{Проста збіжність: }
\begin{definition}
    $(f_n)_{n \in \N}$ збігається просто до $f$ якщо $\forall x_0 \in X$, $f_n(x_0) \xrightarrow[n \to \infty]{} f(x_0)$ (не походить від норми).
\end{definition}
\subsection{Рівномірна збіжність: }
\begin{definition}
    $f \in B(X, \R)$ якщо $\sup_{x \in X} |f(x)| = \|f\|_{\infty} < \infty$ ($f$ обмежена на $X$). \\
    Рівномірна збіжність: $\forall \varepsilon > 0, \exists N \in \N$ tq $\forall n \ge N \forall x \in X |f_n(x) - f(x)| < \varepsilon$ еквівалентно
    \[
    \forall \varepsilon > 0 \exists N \in \N \text{ tq } \forall n \ge N, \|f_n - f\|_{\infty} < \varepsilon
    \] 
    $f_n \to f$ в $(B(X, \R), \| \cdot \|_{\infty})$
\end{definition}
\begin{definition}
    Рівномірна границя неперервних функцій: $X = [a, b]$, $\mathcal{C}([a, b], \R) \subset B([a, b], \R)$(підпростори векторів). $\mathcal{C}([a, b], \R)$ є замкнутим в $(B([a, b], \R), \| \cdot \|_{\infty})$
\end{definition}
\subsection{Ряди зі значеннями в нормованому векторному просторі.}
\begin{definition}
     Нехай $(E, \| \cdot \|_{\infty})$ e.v.n\footnote{нормований векторний простір}, $(u_n)_{n \in N}$ послідовність в $E$. Ряд $\sum u_n$ збігається в $(E, \| \cdot \|)$ якщо послідовність $S_N = \sum_{n=0}^{N} u_n$ збігається в $(E, \| \cdot \|)$. $\lim_{N \to \infty} S_N$ позначається $\sum_{n=0}^{\infty} u_n (\in E)$
\end{definition}
\begin{remark}
   Якщо $\sum u_n$ і $\sum v_n$ збігаються, тоді
   \begin{itemize}
       \item $\sum u_n + v_n$ збігається і $\sum \lambda u_n$ збігається
       \item $\sum_{n=0}^{\infty} u_n + v_n = \sum_{n=0}^{\infty} u_n + \sum_{n=0}^{\infty} v_n$ 
       \item $\sum_{n=0}^{\infty} \lambda u_n = \lambda \sum_{n=0}^{\infty} u_n$
   \end{itemize}
\end{remark}

\subsection{Нормальна збіжність}
\begin{definition}
   $\sum u_n$ збігається нормально в $(E, \| \cdot \|)$ якщо $\sum \|u_n\|$ збігається в $\R$. 
\end{definition}
\begin{eg}
   $E = \R$, $\|x\| = |x|$. cv. нормальна = cv. абсолютна ($\sum u_n$ збігається) 
\end{eg}
\begin{eg}
   $\sum u_n$ може збігатися без нормальної збіжності, як: $u_n = \frac{(-1)^n}{n}$ 
\end{eg}

\begin{theorem}
    Якщо $(E, \| \cdot \|)$ є повним, кожен нормально збіжний ряд є збіжним і
     \[
    \|\sum_{n=0}^{\infty} u_n\| \le \sum_{n=0}^{\infty} \|u_n\|
    \] 
\end{theorem}
\begin{preuve}
   $S_n = \sum_{k=0}^{n} u_k$ і $T_n = \sum_{k=0}^{n} \|u_k\|$ 

   \begin{align*}
       n > p \quad \|S_n - S_p\| = \|\sum_{k = p+1}^{n} u_k\| \le \sum_{k=p+1}^{n} \|u_k\| = T_n - T_p = |T_n - T_p|
   \end{align*}
    $(T_n)$ збігається в $\R$, тому $(T_n)$ Коші:
     \[
    \forall \varepsilon > 0, \exists N \text{ tq } \forall n > p \ge N |T_n - T_p| \le \varepsilon
    \] 
    тому $(S_n)$ Коші в $(E, \| \cdot \|)$. \underline{$E$ повний:} $(S_n)$ збігається до $S \in E$.
\end{preuve}
\section{Лінійні неперервні відображення}
Для будь-якої секції $B_E$ позначає кулю \underline{замкнену}!
\par
Нехай $E, F$ 2 нормовані векторні простори з $\| \cdot \|_{E}$ та $\| \cdot \|_{F}$ відповідними нормами, 
\begin{itemize}
    \item $A \in \mathcal{L}(E, F)$
    \item $\lambda A \in \mathcal{L}(E, F)$ і $\lambda Ax = \lambda(Ax)$
    \item $A + B \in \mathcal{L}(E, F)$ і $(A + B)x = Ax + Bx$
    \item $0x = 0_F$ $\forall x \in E$
\end{itemize}
\[
    \mathcal{L}(E) = \mathcal{L}(E, E)
\] 
\begin{itemize}
    \item $(AB)x = A(Bx)$ де $AB = A \circ B$
    \item $(\lambda A)B = \lambda (AB)$
    \item $A(B + C) = AB + AC$
    \item $(A + B)C = AC + BC$
    \item $0A = 0$ 
    \item $AB \neq BA$ (в загальному)
    \item $A(BC) = (AB)C$
\end{itemize}

\begin{theorem}
    Нехай $A \in \mathcal{L}(E, F)$. Наступні властивості еквівалентні:
    \begin{enumerate}
        \item $A: E \to F$ є неперервним
        \item $A$ є неперервним в $0_E$
        \item $\exists C \ge 0$ така що 
            \[
            \|Ax\|_F \le C\|x\|_E \quad \forall x \in E
            \] 
            це називається, що $A$ є обмеженим
        \item $A$ є обмеженим на $B_E(0, R)$ $\forall R > 0$
    \end{enumerate}
    Кажуть, що $A$ є обмеженим (якщо $A$ є неперервним і лінійним)
\end{theorem}
\begin{preuve}
    \begin{itemize}
        \item $1) \implies 2)$ : очевидно
        \item $2) \implies 3)$ :

            \begin{itemize}
                \item 
                    \underline{Hyp:} $\forall \varepsilon >0, \exists \delta > 0$ tq $\|x - 0_E\|_E \le \delta \implies \|Ax - A0_E\|_F \le \varepsilon$ $\|x\|_E \le \delta \implies \|Ax\|_F \le \varepsilon$
                \item $\varepsilon = 1 \exists \delta > 0$ tq $\|x\|_E \le \delta \implies \|Ax\|_F \le 1$
                \item Нехай $ x \in E$ і $x \neq 0_E$
                \item $y = \frac{\delta}{\|x\|_E}x$ тому $\|y\|_E = \delta$  $\implies\|Ay\|_F \le 1$
                \item $Ay = \frac{\delta}{\|x\|_{E}}Ax$ і $A$ лінійний
                \item  $\|Ay\|_{F} = \frac{\delta}{\|x\|_E}\|Ax\|_F \le 1 \implies \|Ax\|_F \le \frac{1}{\delta}\|x\|_E$
            \end{itemize}
        \item $3) \implies 1)$ 
            \begin{itemize}
                \item Я фіксую $x_0 \in E$. треба побачити: $A$ неперервний в  $x_0$?
                \item $\|Ax - Ax_0\|_F = \|A(x - x_0)\|_F \le C\|x - x_0\|_E$
                \item Тому якщо $\|x - x_0\|_E \le \frac{\varepsilon}{c} = \delta(\varepsilon)$, $\|Ax - Ax_0\|_F \le \varepsilon$
            \end{itemize}
    \end{itemize}
\end{preuve}

\begin{notation}
   \[
       B(E, F) = \{ A \in \mathcal{L}(E, F): A \text{ continue } \}
   \]  
   \[
   B(E, E) = B(E)
   \] 
\end{notation}
\begin{lemma}
   Якщо $E$ має скінченну розмірність, тоді
   \[
       \mathcal{L}(E, F) = B(E, F)
   \] 
   Це невірно, якщо $\dim E = \infty$
\end{lemma}
\begin{preuve}
    $(e_1, \ldots, e_n)$ база $E$. На  $E$ всі норми еквівалентні.  
    \begin{itemize}
        \item $\|x\|_E$ задана норма.        
        \item $\|x\|_{\infty} = \max_{1 \le i \le n} |x_i|$ 
    \end{itemize}
    де $x = \sum_{i=1}^{n} x_ie_i$ 
    \[
    \|Ax\|_F = \|\sum_{i=1}^{n} x_iAe_i\| = \sum_{i=1}^{n} |x_i|\|Ae_i\|_F
    \] 
    \[
    \|Ax\|_F \le \|x\|_{\infty} \times \sum_{i=1}^{n} \|Ae_i\|_F = C\|x\|_{\infty}
    \] 
    ($\|x\|_{\infty}\| \le C'\|x\|_{E}$). Тому: $\|Ax\|_{F} \le CC'\|x\|_E$. Тоді: $A \in B(E, F)$
\end{preuve}

\subsection{Норма на $B(E, F)$}

\begin{theorem}
    Нехай $A \in B(E, F)$, позначимо $\displaystyle \|A\| = \sup_{x \in E, \|x\|_E \le 1} \|Ax\|_F = \sup_{x \in B_E(0, 1)} \|Ax\|_{F}$

    \begin{enumerate}
        \item $\| \cdot \|$ є нормою на $B(E, F)$, що називається рівномірною нормою.
        \item Маємо: $\|Ax\|_F \le \|A\|\|x\|_{E} \quad \forall x \in E$
        \item $\|A\| = $ найменша константа $C$ така що $\|Ax\|_{F} \le C\|x\|_{E} \quad \forall x \in E$
    \end{enumerate}
\end{theorem}
\begin{remark}
    \begin{enumerate}
        \item Можна писати $\|A\|_{B(E, F)}$ замість $\|A\|$ 
        \item Іноді можна зустріти $|||A|||$ для $\|A\|$ 
        \item Нехай $I^+ = $ множина $C \ge 0$ така що $\|Ax\|_{F} \le C\|x\|_{E} \quad \forall x \in E$. \\
            $I^+ \neq \O$ (оскільки $A \in B(E, F)$) і $I^+ \subset [0, +\infty[$. $(2)$ і $(3)$ кажуть, що $\|A\|$ є найменшим елементом $I^+$
             \[
            \inf I^+ = \min I^+ = \|A\|
            \] 
    \end{enumerate}
\end{remark}
\begin{preuve}
   \begin{enumerate}
       \item $A \in B(E, F) \iff \sup_{x \in B_{E}(0, 1)} \|Ax\|_{F} < \infty \iff \|A\|$ добре визначена.
           \begin{align*}
               \|(A + B)x\|_F = \|Ax + Bx\|_F \le \|Ax\|_F + \|Bx\|_F\\ \implies \sup_{x \in B_E(0, 1)} \|(A + B)x\|_F \le \sup_{x \in B_E(0, 1)} \|Ax\|_F + \sup_{x \in B_E(0, 1)} \|Bx\|_F
           \end{align*}
           \[
               \|A + B\| \le \|A\| + \|B\| \text{ та } A, B \in B(E, F) \implies A + B \text{ також }
           \] 
           \[
           \|\lambda A\| = |\lambda|\|A\| \text{ та } A \in B(E, F) \implies \lambda A \text{ також }
           \] 
           Якщо $\|A\| = 0$, тоді $\|Ax\|_F = 0 \forall x \in B_E(0, 1) \implies Ax = 0_F \forall x \in B_E(0, 1)$
           \[
           Ax = \|x\|_E A \frac{x}{\|x\|_E}
           \] 
           \[
           Ax = 0_F \forall x \in E \implies A = 0_{L(E, F)}
           \] 
           \[
           C \in I^+ \text{ якщо } \|Ax\|_F \le C\|x\|_E \quad \forall x \in E
           \] 
           \[
           \|A\| \in I^+ \implies \|Ax\|_F \le \|A\|\|x\|_E \forall x
           \] 
           \begin{itemize}
               \item Очевидно якщо $x = 0_E$.
               \item Якщо $x \neq 0_E$, $y = \frac{x}{\|x\|_E} \in B_E(0, 1)$ отже
                   \[
                       \|Ay\|_F = \frac{1}{\|x\|_E}\|Ax\|_F \le \|A\| \implies \|Ax\|_F \le \|A\|\|x\|_E

                   \] 
                   Нехай $C \in I^+$ тому  
                   \[
                   \|Ax\|_F \le C\|x\|_E
                   \] 
                   тому $\|Ax\|_F \le C \quad \forall x \in B_E(0,1)$, тому $\|A\| \le C$, тоді 
                   \[
                       \|A\| = \min I^+ = \text{ "найкраща константа $C$"}
                   \] 
           \end{itemize}
   \end{enumerate} 
\end{preuve}
\begin{eg}
    $E = \mathcal{C}([a, b], \R)$, $\|f\|_{\infty} = \sup_{x \in [a, b]} |f(x)|$, $F = \R$, $u \in \mathcal{C}([a, b], \R)$
    \begin{align*}
        A: E &\longrightarrow F \\
        f &\longmapsto A(f) = \int_{{a}}^{{b}} {f(x)u(x)} \: d{x} {}
    .\end{align*}
    \underline{$A$ обмежена}: потрібно побачити: $\exists C \ge 0$ така що  
    \[
        \left| \int_{{a}}^{{b}} {f(x)u(x)} \: d{x} {} \right| \le C \sup_{x \in [a, b]} |f(x)|
    \] 
    ?
    \begin{align*}
        \left| \int_{{a}}^{{b}} {f(x)u(x)} \: d{x} {} \right| \le \int_{{a}}^{{b}} {|f(x)| |u(x)|} \: d{x} {} \le \int_{{a}}^{{b}} {\|f\|_{\infty}|u(x)|} \: d{x} {= \|f\|_{\infty} \int_{{a}}^{{b}} {|u(x)|} \: d{x} {}}
    \end{align*}
    \[
    C = \int_{{a}}^{{b}} {|u(x)|} \: d{x} \text{ підходить }
    \] 
    (Фактично $\|A\| = \int_{{a}}^{{b}} {|u(x)|} \: d{x} {}$). $E = \mathcal{C}^1([0,1], \R)$ muni de $\|f\|_{\infty}$, $F = \R$, $Af = f'(0)$ linéaire mais pas continue. On construit une suite  $(f_n)$ dans  $E$ telle que  $\|f_n\|_E \xrightarrow[n \to  \infty]{} 0$ mais $\|Af_n\|_F \not\to 0$
    \[
    f_n(x) = \frac{1}{n}\sin(nx)
    \] 
\end{eg}

